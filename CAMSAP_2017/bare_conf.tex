
%% bare_conf.tex
%% V1.3
%% 2007/01/11
%% by Michael Shell
%% See:
%% http://www.michaelshell.org/
%% for current contact information.
%%
%% This is a skeleton file demonstrating the use of IEEEtran.cls
%% (requires IEEEtran.cls version 1.7 or later) with an IEEE conference paper.
%%
%% Support sites:
%% http://www.michaelshell.org/tex/ieeetran/
%% http://www.ctan.org/tex-archive/macros/latex/contrib/IEEEtran/
%% and
%% http://www.ieee.org/

%%*************************************************************************
%% Legal Notice:
%% This code is offered as-is without any warranty either expressed or
%% implied; without even the implied warranty of MERCHANTABILITY or
%% FITNESS FOR A PARTICULAR PURPOSE! 
%% User assumes all risk.
%% In no event shall IEEE or any contributor to this code be liable for
%% any damages or losses, including, but not limited to, incidental,
%% consequential, or any other damages, resulting from the use or misuse
%% of any information contained here.
%%
%% All comments are the opinions of their respective authors and are not
%% necessarily endorsed by the IEEE.
%%
%% This work is distributed under the LaTeX Project Public License (LPPL)
%% ( http://www.latex-project.org/ ) version 1.3, and may be freely used,
%% distributed and modified. A copy of the LPPL, version 1.3, is included
%% in the base LaTeX documentation of all distributions of LaTeX released
%% 2003/12/01 or later.
%% Retain all contribution notices and credits.
%% ** Modified files should be clearly indicated as such, including  **
%% ** renaming them and changing author support contact information. **
%%
%% File list of work: IEEEtran.cls, IEEEtran_HOWTO.pdf, bare_adv.tex,
%%                    bare_conf.tex, bare_jrnl.tex, bare_jrnl_compsoc.tex
%%*************************************************************************

% *** Authors should verify (and, if needed, correct) their LaTeX system  ***
% *** with the testflow diagnostic prior to trusting their LaTeX platform ***
% *** with production work. IEEE's font choices can trigger bugs that do  ***
% *** not appear when using other class files.                            ***
% The testflow support page is at:
% http://www.michaelshell.org/tex/testflow/



% Note that the a4paper option is mainly intended so that authors in
% countries using A4 can easily print to A4 and see how their papers will
% look in print - the typesetting of the document will not typically be
% affected with changes in paper size (but the bottom and side margins will).
% Use the testflow package mentioned above to verify correct handling of
% both paper sizes by the user's LaTeX system.
%
% Also note that the "draftcls" or "draftclsnofoot", not "draft", option
% should be used if it is desired that the figures are to be displayed in
% draft mode.
%
\documentclass[conference]{IEEEtran}
% Add the compsoc option for Computer Society conferences.
%
% If IEEEtran.cls has not been installed into the LaTeX system files,
% manually specify the path to it like:
% \documentclass[conference]{../sty/IEEEtran}



%packages
\usepackage{amsmath}
\usepackage{amssymb,amstext,amsfonts,bm}

% Acronimies
\usepackage[acronyms,nonumberlist]{glossaries}

% Figure representation
\usepackage{graphicx,subfigure}

% Algorithm table
\usepackage{algorithm}

\input{../../latex-aux/mymath}

% \newacronym{mse}{MSE}{mean square error}%
\newacronym{mmse}{MMSE}{minimum mean square error}%
\newacronym{cs}{CS}{compressive sensing}
\newacronym{rip}{RIP}{restricted isometry property}
\newacronym{mimo}{MIMO}{multiple-input multiple-output}
\newacronym{ls}{LS}{least square}
\newacronym{dft}{DFT}{discrete Fourier transmform}
\newacronym{omp}{OMP}{orthogonal matching-pursuit}
\newacronym{t-omp}{T-OMP}{orthogonal matching-pursuit}
\newacronym{fdd}{FDD}{Frequency Division Multiplexing}
\newacronym{admm}{ADMM}{alternate direction method of multipliers}
\newacronym{snr}{SNR}{signal-noise ratio}
\newacronym{crlb}{CRLB}{Cramer-Rao lower bound}
\newacronym{mvu}{MVU}{minimum variance and unbiased}
\newacronym{pdf}{PDF}{probability density function}
\newacronym{csi}{CSI}{channel state information}
\newacronym{3G}{3G}{Third generation}
\newacronym{AN}{AN}{accesses node}
\newacronym{BS}{BS}{base station}
\newacronym{DJTF}{DJTF}{discrete joint time-frequency}
\newacronym{DoA}{DoA}{direction of arrival}
\newacronym{DoD}{DoD}{direction of departure}
\newacronym{FP}{FP}{frequency dependent precoder}
\newacronym{HOSVD}{HOSVD}{High-order singular values decomposition}
\newacronym{IRC}{IRC}{interference rejection combining}
\newacronym{LOS}{LOS}{line-of-sight}
\newacronym{LTE}{LTE}{Long Term Evolution}
\newacronym{LTV}{LTV}{linear time variant}
\newacronym{MU}{MU}{mobile unit}
\newacronym{NLOS}{NLOS}{non-line-of-sight}
\newacronym{NMSE}{NMSE}{Normalized Mean Square Error}
\newacronym{OFDM}{OFDM}{Orthogonal Frequency Division Multiplexing}
\newacronym{MRC}{MRC}{maximum ratio combiner}
\newacronym{PCWP}{PCWP}{phase-constrained wideband precoder}
\newacronym{QoS}{QoS}{Quality of Service}
\newacronym{RIP}{RIP}{Restricted Isometry Property}
\newacronym{SFT}{SFT}{scattering function tensor}
\newacronym{SINR}{SINR}{signal-interference-noise ratio}
\newacronym{SVD}{SVD}{singular value decomposition}
\newacronym{TDD}{TDD}{time-division duplexing}
\newacronym{TF}{TF}{time-frequency}
\newacronym{UE}{UE}{user equipment}
\newacronym{WLAN}{WLAN}{Wireless Local Area Network}
\newacronym{WP}{WP}{wideband precoder}
\newacronym{WSSUS}{WSSUS}{wide-sense stationary uncorrelated scattering}
\newacronym{ZF}{ZF}{zero-forcing}
\newacronym{FDD}{FDD}{frequency division duplexing}
\newacronym{ADC}{ADC}{analog-digital converter}
\newacronym{HB}{HB}{hybrid beamforming}
\newacronym{DB}{DB}{digital beamforming}
\newacronym{RF}{RF}{radio frequency}
\newacronym{ALSP}{ALSP}{alternating least-square with projection}
\newacronym{KKT}{KKT}{Karush Kuhn Tucker}
\newacronym{mmWave}{mmWave}{milimeter wave}
\newacronym{DoF}{DoF}{degree of freedom}
\newacronym{UL}{UL}{uplink}
\newacronym{MRT}{MRT}{maximum ratio transmission}
\newacronym{DAC}{DAC}{digital to analog converter}
\newacronym{AoA}{AoA}{angle of arrival}
\newacronym{SC}{SC}{small cells}
\newacronym{HetNet}{HetNet}{heterogeneous network}
\newacronym{D2D}{D2D}{device-to-device}
\newacronym{DL}{DL}{downlink}
\newacronym{PARAFAC}{PARAFAC}{Parallel Factors}
\newacronym{RB}{RB}{resource block}






\begin{document}
%
% paper title
% can use linebreaks \\ within to get better formatting as desired
\title{Channel Estimation of Frequency Selective Channels Using Sparse Tensor Processing}


% author names and affiliations
% use a multiple column layout for up to three different
% affiliations
\author{\IEEEauthorblockN{Daniel C. Ara\'ujo}
\IEEEauthorblockA{Research Group of Wireless Communication (GTEL) \\Department of Teleinformatics\\
Federal University of Ceará\\
Cear\'a, Fortaleza 60020-181\\
Email: araujo@gtel.ufc.br}
\and
\IEEEauthorblockN{Andr\'e Lima F. de Almeida}
\IEEEauthorblockA{Research Group of Wireless Communication (GTEL) \\Department of Teleinformatics\\
Federal University of Ceará\\
Cear\'a, Fortaleza 60020-181\\
Email: andre@gtel.ufc.br}}

% conference papers do not typically use \thanks and this command
% is locked out in conference mode. If really needed, such as for
% the acknowledgment of grants, issue a \IEEEoverridecommandlockouts
% after \documentclass

% for over three affiliations, or if they all won't fit within the width
% of the page, use this alternative format:
% 
%\author{\IEEEauthorblockN{Michael Shell\IEEEauthorrefmark{1},
%Homer Simpson\IEEEauthorrefmark{2},
%James Kirk\IEEEauthorrefmark{3}, 
%Montgomery Scott\IEEEauthorrefmark{3} and
%Eldon Tyrell\IEEEauthorrefmark{4}}
%\IEEEauthorblockA{\IEEEauthorrefmark{1}School of Electrical and Computer Engineering\\
%Georgia Institute of Technology,
%Atlanta, Georgia 30332--0250\\ Email: see http://www.michaelshell.org/contact.html}
%\IEEEauthorblockA{\IEEEauthorrefmark{2}Twentieth Century Fox, Springfield, USA\\
%Email: homer@thesimpsons.com}
%\IEEEauthorblockA{\IEEEauthorrefmark{3}Starfleet Academy, San Francisco, California 96678-2391\\
%Telephone: (800) 555--1212, Fax: (888) 555--1212}
%\IEEEauthorblockA{\IEEEauthorrefmark{4}Tyrell Inc., 123 Replicant Street, Los Angeles, California 90210--4321}}




% use for special paper notices
%\IEEEspecialpapernotice{(Invited Paper)}




% make the title area
\maketitle


\begin{abstract}
%\boldmath
The abstract goes here.
\end{abstract}
% IEEEtran.cls defaults to using nonbold math in the Abstract.
% This preserves the distinction between vectors and scalars. However,
% if the conference you are submitting to favors bold math in the abstract,
% then you can use LaTeX's standard command \boldmath at the very start
% of the abstract to achieve this. Many IEEE journals/conferences frown on
% math in the abstract anyway.

% no keywords




% For peer review papers, you can put extra information on the cover
% page as needed:
% \ifCLASSOPTIONpeerreview
% \begin{center} \bfseries EDICS Category: 3-BBND \end{center}
% \fi
%
% For peerreview papers, this IEEEtran command inserts a page break and
% creates the second title. It will be ignored for other modes.
\IEEEpeerreviewmaketitle



\section{Introduction}


\textit{Notation}: A scalar is denoted in italic, e.g. $a$. A column vector is a
bold lowercase letter, e.g. $\bfa$ whose $i$th entry is $\bfa(i)$. A matrix is
denoted by a bold uppercase letter, e.g. $\bfA$ with $(i,j)$th entry
$\bfA(i,j)$; $\bfA(:,j)$ is the $j$th column of $\bfA$. A third-order tensor is
a caligrafic uppercase letter, e.g. $\calA$ with $(i,j,k)$th entry
$\calA(i,j,k)$. The $\Vector{.}$ operator stacks the columns of its argument
into one big column  vector; $\otimes$ stands for the Kronecker product and
$\odot$ is the Khatri-Rao product (column-wise Kronecker product) \cite{Sidiropoulos:2000,Sidiropoulos:2012}. 
\section{System Model}

Consider a single user \gls{mimo} system with tranmitter and receiver equipped with
$N_t$  and  $N_r$ antennas, respectively.


Consider a $N_r \times N_t$ channel matrix between transmitter and receiver
composed by a summation of  $N_c$ delay tap matrices $\bfH_d$,
$d=\{0,1, \ldots, N_c-1\}$. The variable $\rho$ denotes the average received power and $\bfz_n \sim $
$\mathcal{N}(0,\sigma ^2\bfI)$ the circularly symmetric complex Gaussian
distributed noise vector, therefore, the received signal is expressed as

\begin{equation}
  \label{eq:rx_signal}
   \bfr_{l,n} = \sqrt{\rho} \sum_{d=0}^{N_c-1} \bfH_d\bff_{l}s_{n-d} + \bfz_{l,n},
\end{equation}
where $s_{n}$ is the $n$th non-zero of the training frame of length $N + N_c-1$
\begin{equation}
  \label{eq:frame}
  \bfs = [0, \ldots, 0,  s_1, s_2, \ldots, s_N ].
\end{equation}
At the receiver, and a $\textrm{RF}$  combiner $\bfw_k$ is applied over the
training frame, so that the combiner output is expressed as
\begin{equation}
  \label{eq:output_combiner}
   y_{k,l,n} = \sqrt{\rho}\bfw^H_k \sum_{d=0}^{N_c-1} \bfH_d\bff_{l}s_{n-d} + z_{k,l,n},
 \end{equation}
 where $z_{k,l,n} = w^H_k\bfz_{l,n}$
The output signal can described in term of tensor notation

\begin{equation}
  \label{eq:output_combiner_tensor}
  y_{k,l,n} =  \calH \times_1 \bfw^H_k \times_2 \bff_{l} \times \bfs_n,
\end{equation}
where $\calH$ is generated from the concatenation in the third dimension of the $N_c$ delay taps.
Assuming a collection  of transmitter and receiver beams, i.e.
$l=\{1,\ldots , L \}$ and $k=\{1,\ldots , K \}$, respectively, and the $N$ time
instants within the training frame, we can express the signal model as the
third-order tensor

\begin{equation}
  \label{eq:tensor_model}
  \calY =  \calH \times_1 \bfW \times_2 \bfF \times_3 \bfS + \calZ \times_1 W,
\end{equation}
where $\bfS$ is toeplitz because there is a convolution operation over the third dimension.


\section{Conclusion}
The conclusion goes here.

\section*{Acknowledgment}

\bibliographystyle{IEEEtran}
\bibliography{../../latex-aux/reference}

\end{document}


